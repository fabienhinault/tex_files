\def\\#1{{\tt \char"5C #1}}

We still haven’t discussed the special trick that allows the final line of a
paragraph to be shorter than the others.
Just before TEX begins to choose
breakpoints, it does two important things: (1) If the final item of the current horizontal
list is glue, that glue is discarded. 
(The reason is that a blank space often gets into a
token list just before \\par or just before \$\$, 
and this blank space should not be part
of the paragraph.) 
(2) Three more items are put at the end of the current horizontal
list: \\{penalty10000} (which prohibits a line break); \\{hskip\\parfillskip} (which adds
“finishing glue” to the paragraph);
and \\penalty-10000 (which forces the final break).
Plain TEX sets \\parfillskip=0pt plus1fil,
so that the last line of each paragraph will
be filled with white space if necessary;
but other settings of \\parfillskip are appropriate in special applications.
For example, the present paragraph ends flush with the
right margin, because it was typeset with \\parfillskip=0pt; the author didn’t have to
rewrite any of the text in order to make this possible, since a long paragraph generally
allows so much flexibility that a line break can be forced at almost any point. You
can have some fun playing with paragraphs, because the algorithm for line breaking
occasionally appears to be clairvoyant. Just write paragraphs that are long enough.

\it
We still haven’t discussed the special trick that allows the final line of a
paragraph to be shorter than the others.
Just before TEX begins to choose
breakpoints, it does two important things: (1) If the final item of the current horizontal
list is glue, that glue is discarded. 
(The reason is that a blank space often gets into a
token list just before \\par or just before \$\$, 
and this blank space should not be part
of the paragraph.) 
(2) Three more items are put at the end of the current horizontal
list: \\{penalty10000} (which prohibits a line break); \\{hskip\\parfillskip} (which adds
“finishing glue” to the paragraph);
and \\penalty-10000 (which forces the final break).
Plain TEX sets \\parfillskip=0pt plus1fil,
so that the last line of each paragraph will
be filled with white space if necessary;
but other settings of \\parfillskip are appropriate in special applications.
For example, the present paragraph ends flush with the
right margin, because it was typeset with \\parfillskip=0pt; the author didn’t have to
rewrite any of the text in order to make this possible, since a long paragraph generally
allows so much flexibility that a line break can be forced at almost any point. You
can have some fun playing with paragraphs, because the algorithm for line breaking
occasionally appears to be clairvoyant. Just write paragraphs that are long enough.

1pt

\leftskip=-1pt
\rightskip=1pt
We still haven’t discussed the special trick that allows the final line of a
paragraph to be shorter than the others.
Just before TEX begins to choose
breakpoints, it does two important things: (1) If the final item of the current horizontal
list is glue, that glue is discarded. 
(The reason is that a blank space often gets into a
token list just before \\par or just before \$\$, 
and this blank space should not be part
of the paragraph.) 
(2) Three more items are put at the end of the current horizontal
list: \\{penalty10000} (which prohibits a line break); \\{hskip\\parfillskip} (which adds
“finishing glue” to the paragraph);
and \\penalty-10000 (which forces the final break).
Plain TEX sets \\parfillskip=0pt plus1fil,
so that the last line of each paragraph will
be filled with white space if necessary;
but other settings of \\parfillskip are appropriate in special applications.
For example, the present paragraph ends flush with the
right margin, because it was typeset with \\parfillskip=0pt; the author didn’t have to
rewrite any of the text in order to make this possible, since a long paragraph generally
allows so much flexibility that a line break can be forced at almost any point. You
can have some fun playing with paragraphs, because the algorithm for line breaking
occasionally appears to be clairvoyant. Just write paragraphs that are long enough.

10pt

\leftskip=-10pt
\rightskip=10pt
We still haven’t discussed the special trick that allows the final line of a
paragraph to be shorter than the others.
Just before TEX begins to choose
breakpoints, it does two important things: (1) If the final item of the current horizontal
list is glue, that glue is discarded. 
(The reason is that a blank space often gets into a
token list just before \\par or just before \$\$, 
and this blank space should not be part
of the paragraph.) 
(2) Three more items are put at the end of the current horizontal
list: \\{penalty10000} (which prohibits a line break); \\{hskip\\parfillskip} (which adds
“finishing glue” to the paragraph);
and \\penalty-10000 (which forces the final break).
Plain TEX sets \\parfillskip=0pt plus1fil,
so that the last line of each paragraph will
be filled with white space if necessary;
but other settings of \\parfillskip are appropriate in special applications.
For example, the present paragraph ends flush with the
right margin, because it was typeset with \\parfillskip=0pt; the author didn’t have to
rewrite any of the text in order to make this possible, since a long paragraph generally
allows so much flexibility that a line break can be forced at almost any point. You
can have some fun playing with paragraphs, because the algorithm for line breaking
occasionally appears to be clairvoyant. Just write paragraphs that are long enough.

\leftskip=0pt
\rightskip=0pt
\rm
We still haven’t discussed the special trick that allows the final line of a
paragraph to be shorter than the others.
Just before TEX begins to choose
breakpoints, it does two important things: (1) If the final item of the current horizontal
list is glue, that glue is discarded. 
(The reason is that a blank space often gets into a
token list just before \\par or just before \$\$, 
and this blank space should not be part
of the paragraph.) 
(2) Three more items are put at the end of the current horizontal
list: \\{penalty10000} (which prohibits a line break); \\{hskip\\parfillskip} (which adds
“finishing glue” to the paragraph);
and \\penalty-10000 (which forces the final break).
Plain TEX sets \\parfillskip=0pt plus1fil,
so that the last line of each paragraph will
be filled with white space if necessary;
but other settings of \\parfillskip are appropriate in special applications.
For example, the present paragraph ends flush with the
right margin, because it was typeset with \\parfillskip=0pt; the author didn’t have to
rewrite any of the text in order to make this possible, since a long paragraph generally
allows so much flexibility that a line break can be forced at almost any point. You
can have some fun playing with paragraphs, because the algorithm for line breaking
occasionally appears to be clairvoyant. Just write paragraphs that are long enough.


\bye

