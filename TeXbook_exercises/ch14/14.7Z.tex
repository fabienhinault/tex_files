{\obeylines\smallskip
Roses are red,
\indent Violets are blue;
Rhymes can be typeset
\indent With boxes and glue.
\smallskip}

{\obeylines\smallskip
Roses are red,
\quad Violets are blue;
Rhymes can be typeset
\quad With boxes and glue.
\smallskip}

{\obeylines\smallskip
Roses are red,
\indent\indent  Violets are blue;
Rhymes can be typeset
\indent\indent  With boxes and glue.
\smallskip}

\vbox{
\hbox{Roses are red,}
\hbox{\indent Violets are blue;}
\hbox{Rhymes can be typeset}
\hbox{\indent With boxes and glue.}
}

The second and fourth lines are indented by an additional “quad” of space,
i.e., by one extra em in the current type style. (The control sequence quad does an
hskip; when TEX is in vertical mode, hskip begins a new paragraph and puts glue
after the indentation.) If indent had been used instead, those lines wouldn’t have
been indented any more than the first and third, because indent is implicit at the
beginning of every paragraph. Double indentation on the second and fourth lines could
have been achieved by ‘indent indent’.


\bye
