\def\centerlinea#1{\line{\hfil#1\hfil}}
\def\centerlineb#1{\line{\hfill#1\hfill}}
\def\centerlinec#1{\line{\hss#1\hss}}

With centerlinec, the line size can be less than \#1's size.
In centerlinea, a hfil contained in \#1 will stretch, but not in centerlineb.

Answer:

The first two give an “overfull box” if the argument doesn’t fit on a line;
the third allows the argument to stick out into the margins instead. (Plain TEX’s
centerline is centerlinec; the stickout effect shows up in the narrow-column ex-
periment of Chapter 6.) If the argument contains no infinite glue, centerlinea and
centerlineb produce the same effect; but centerlineb will center an argument that
contains ‘fil’ glue.

\bye

