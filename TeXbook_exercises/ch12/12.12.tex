\def\\{\char"5C}
\def\{{\char"7B}
\def\}{\char"7D}
\def\1{{\tt \\box1}}
\def\2{{\tt \\box2}}
\def\3{{\tt \\box3}}}
\def\4{{\tt \\box4}}
{\tt \\baselineskip=9pt minus3fil}

{\tt \\setbox4=\\vbox to4pt\{\\vss\\box1\\moveleft4pt\\box2\\vss\}}

The natural height is 10 pt. The desired height is 4 pt. That's 6 pt too much. These 6 pt are distributed
among the 3 fil of the baseline skip, and the two fil of the two vss. So the baselineskip takes 
$3 \times 6 / 5 = 3.6$ pt. The baselineskip is now 5.4 pt. So \4's height is 6.4 pt, its depth is 0 because
of the ending glue, and its width is 1 pt.

Answer:


The interline glue will be 6 pt minus 3 fil; the final depth will be zero, since
\2 is followed by glue; the natural height is 12 pt; and the shrinkability is 5 fil. So
\4 will be 4 pt high, 0 pt deep, 1 pt wide, and it will contain five items: \\vskip
-1.6pt, \1, \\vskip1.2pt, \\moveleft4pt\2, \\vskip-1.6pt. Starting at the ref-
erence point of \4, you get to the reference point of \1 by going up 4.6 pt, or
to the reference point of \2 by going up .4 pt and left 4 pt. (For example, you go
up 4 pt to get to the upper left corner of \4; then down −1.6 pt, i.e., up 1.6 pt, to
get to the upper left corner of \1; then down 1 pt to reach its reference point. This
problem is clearly academic, since it’s rather ridiculous to include infinite shrinkability
in the baselineskip.)

\bye
