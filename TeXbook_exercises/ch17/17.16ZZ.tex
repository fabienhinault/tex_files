\def\sqboxraw{\hbox to3pt {\vrule \vbox to3pt{\hrule \vfil \hfil \hrule} \vrule}}
\def\sqbox{\vcenter to3pt {\sqboxraw}}
\def\sqone{\hbox to3pt {\vrule height3pt \hfil \vrule height3pt}}
% \def\sqtwo{\hbox to3pt {\vrule height3pt \hrule\vrule height3pt}}
% You can't use `\hrule' here except with leaders.
\def\sqtwo{\hbox to3pt {\vrule height3pt \leaders\hrule\hfil\vrule height3pt}}
% \def\sqthree{\hbox to3pt {\vrule height3pt \vbox to3pt{{\leaders\hrule\hfil}\vfil{\leaders\hrule\hfil}}\vrule height3pt}}
\def\sqa{\hbox to3pt {\vrule height3pt \vbox to3pt{\hrule width2.2pt\vfil\hrule width2.2pt}\vrule}}

\def\sqb{\vcenter to3pt{\hrule width3pt \hbox to 3pt{\vrule height2.2pt \hfil \vrule height2.2pt}\hrule width3pt}}
\def\sqdisplaytext{\vcenter to3pt{\hrule width3pt \hbox to 3pt{\vrule height2.2pt \hfil \vrule height2.2pt}\hrule width3pt}}
\def\sqscript{\vcenter to2.1pt{\hrule height0.3pt width2.1pt \hbox to 2.1pt{\vrule width0.3pt height1.5pt \hfil \vrule width0.3pt height1.5pt}\hrule height0.3pt width2.1pt}}
\def\sqscriptscript{\vcenter to1.5pt{\hrule width3pt width1.5pt \hbox to 1.5pt{\vrule width0.3pt height.9pt \hfil \vrule width0.3pt height.9pt}\hrule width3pt width1.5pt}}

\def\square{\mathchoice{\sqdisplaytext}{\sqdisplaytext}{\sqscript}{\sqscriptscript}}

$\square + x_{\square}^{\square} + x^{x^{\square}_{\square}}$

$\sqb + \sqb + x$ \sqa

\sqone

\sqtwo

% \sqfour

%\sqthree

\sqboxraw

$\sqbox$

\def\sqr#1#2{{\vcenter{\vbox{\hrule height.#2pt
\hbox{\vrule width.#2pt height#1pt \kern#1pt
\vrule width.#2pt}
\hrule height.#2pt}}}}
\def\squarea{\mathchoice\sqr34\sqr34\sqr{2.1}3\sqr{1.5}3}

$\squarea + x_{\squarea}^{\squarea} + x^{x^{\squarea}_{\squarea}}$

$\square \squarea$

% the given dimensions were those of the inner square.
% It is another interesting exercise to find how to get the outer square to the given dimensions
% without writing the results of subtractions.

\bye

