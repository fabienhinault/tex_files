% https://stackoverflow.com/questions/447617/how-to-modify-the-paper-dimensions-in-plain-tex 
\special{papersize=29.7cm,21cm}
\pdfpageheight 21cm 
\pdfpagewidth 29.7cm 
\advance\hoffset by 1cm
\advance\hoffset by -1in
\advance\voffset by 1cm
\advance\voffset by -1in
\hsize=27.7cm
\vsize=19cm
\nopagenumbers
\font\ss=cmss8 at 21pt
\ss
\font\8=cmss8
\def\math{Math\'ematiques}
\def\hg{Histoire G\'eo}
\def\fran{Fran\c cais}
\def\apfran{AP Fran\c cais}
% set dimen0 to a 5th of (hsize - the place for hours)
\dimen0=\hsize
\setbox0=\hbox{\vrule\800:00 \vrule}
\advance\dimen0 by -\wd0
\divide\dimen0 by 5
\advance\dimen0 by -0.4pt
% a strut for 21pt characters has a height of (21*1.2)/12*8.5 = 17,85pt
% and a depth of (21*1.2)/12*3.5 = 7.35pt
% or 21/10*(7.5+1)...
\def\h#1 #2 {\vbox to 17.85pt{\hbox{\8#1 }\vss\hbox{\lower7.35pt\hbox{\8#2 }}}}
\offinterlineskip
% compute the strut's height and depth
\dimen1=\vsize
\divide\dimen1 by 10 % height + depth
\divide\dimen1 by 120
\dimen2=\dimen1
\multiply\dimen1 by 85 %height
\multiply\dimen2 by 35 %depth
\def\strut{\hbox{\vrule height\dimen1 depth\dimen2 width0pt}} 
\hrule
\halign to \hsize{\vrule#&\8#&&\vrule#&\strut\hbox to \dimen0{\hfil#\hfil}\cr
&&&             LUNDI&&           MARDI&&             MERCREDI&&     JEUDI&&     VENDREDI&\cr
\noalign{\hrule}
&\h08:25 09:20 &&&&              [Vie de classe]&&   \hg&&          Latin&&     Latin&\cr
\noalign{\hrule}
&\h09:23 10:18 &&Arts plastiques&&  Technologies&&      \fran&&        Anglais&&   Physique Chimie&\cr
\noalign{\hrule}
&\h10:33 11:28 &&Allemand&&         \math&&             SVT&&          \math&&     \math&\cr
\noalign{\hrule}
&\h11:35 12:30 &&Musique&&          Anglais&&           &&             Allemand&&  \fran&\cr
\noalign{\hrule}
&&&&&&&&&&&&\cr
\noalign{\hrule}
&\h14:00 14:55 &&Anglais&&          EPS&&               &&             \fran&&      EPS&\cr
\noalign{\hrule}
&\h14:58 15:53 &&\fran&&            \fran&&             &&             \hg&&        EPS&\cr
\noalign{\hrule}
&\h16:10 17:05 &&\hg&&              Projet info num&&   &&             &&           Allemand&\cr}
\hrule

\bye


